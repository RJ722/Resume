%%%%%%%%%%%%%%%%%
% This is an sample CV template created using altacv.cls
% (v1.1.5, 1 December 2018) written by LianTze Lim (liantze@gmail.com). Now compiles with pdfLaTeX, XeLaTeX and LuaLaTeX.
%
%% It may be distributed and/or modified under the
%% conditions of the LaTeX Project Public License, either version 1.3
%% of this license or (at your option) any later version.
%% The latest version of this license is in
%%    http://www.latex-project.org/lppl.txt
%% and version 1.3 or later is part of all distributions of LaTeX
%% version 2003/12/01 or later.
%%%%%%%%%%%%%%%%

%% If you need to pass whatever options to xcolor
\PassOptionsToPackage{dvipsnames}{xcolor}

%% If you are using \orcid or academicons
%% icons, make sure you have the academicons
%% option here, and compile with XeLaTeX
%% or LuaLaTeX.
% \documentclass[10pt,a4paper,academicons]{altacv}

%% Use the "normalphoto" option if you want a normal photo instead of cropped to a circle
% \documentclass[10pt,a4paper,normalphoto]{altacv}

\documentclass[10pt,a4paper,ragged2e]{altacv}

%% AltaCV uses the fontawesome and academicon fonts
%% and packages.
%% See texdoc.net/pkg/fontawecome and http://texdoc.net/pkg/academicons for full list of symbols. You MUST compile with XeLaTeX or LuaLaTeX if you want to use academicons.

% Change the page layout if you need to
\geometry{left=1cm,right=9cm,marginparwidth=6.8cm,marginparsep=1.2cm,top=1.25cm,bottom=1.25cm}

% Change the font if you want to, depending on whether
% you're using pdflatex or xelatex/lualatex
\ifxetexorluatex
  % If using xelatex or lualatex:
  \setmainfont{Carlito}
\else
  % If using pdflatex:
  \usepackage[utf8]{inputenc}
  \usepackage[T1]{fontenc}
  \usepackage[default]{lato}
\fi

% Change the colours if you want to
\definecolor{Mulberry}{HTML}{72243D}
\definecolor{SlateGrey}{HTML}{2E2E2E}
\definecolor{LightGrey}{HTML}{666666}
\colorlet{heading}{Sepia}
\colorlet{accent}{Mulberry}
\colorlet{emphasis}{SlateGrey}
\colorlet{body}{LightGrey}

% Change the bullets for itemize and rating marker
% for \cvskill if you want to
\renewcommand{\itemmarker}{{\small\textbullet}}
\renewcommand{\ratingmarker}{\faCircle}

%% sample.bib contains your publications
\addbibresource{sample.bib}

\usepackage{hyperref}

\begin{document}
\name{Rahul Jha}
\tagline{\faChain \,  \href{https://rj722.github.io}{https://rj722.github.io}}
\photo{2.8cm}{Globe_High}
\personalinfo{%
  % Not all of these are required!
  % You can add your own with \printinfo{symbol}{detail}
  \printinfo{\mailsymbol}{\href{mailto:rahul722j@gmail.com}{rahul722j@gmail.com}}
  % \email{rahul722j@gmail.com}
  \printinfo{\href{https://linkedin.com/in/rj722}{\faLinkedin  \, \underline{LinkedIn}}}{}
  \printinfo{\href{https://github.com/rj722}{\faGithub  \, \underline{GitHub}}}{}
  \phone{\href{tel:+918755930331}{+91-8755-930-331}}
  % \mailaddress{Address, Street, 00000 County}
  % \location{India}
  % \homepage{https://rj722.github.io}
  % \twitter{@twitterhandle}
  % \linkedin{https://linkedin.com/in/RJ722}
  % \github{https://github.com/RJ722}
  %% You MUST add the academicons option to \documentclass, then compile with LuaLaTeX or XeLaTeX, if you want to use \orcid or other academicons commands.
  % \orcid{orcid.org/0000-0000-0000-0000}
}

%% Make the header extend all the way to the right, if you want.
\begin{fullwidth}
\makecvheader
\end{fullwidth}

%% Depending on your tastes, you may want to make fonts of itemize environments slightly smaller
% \AtBeginEnvironment{itemize}{\small}

%% Provide the file name containing the sidebar contents as an optional parameter to \cvsection.
%% You can always just use \marginpar{...} if you do
%% not need to align the top of the contents to any
%% \cvsection title in the "main" bar.
\cvsection[page1sidebar]{Experience}

\cvevent{Data Science Intern}{Humonics Global}{June 2019 -- August 2019}{Gurugram, India}
Automated the approvals of car-insurance claims based on video input.
\begin{itemize}
    \item Employed Differential Evolution for Key Frame Extraction to detect best frames from the given video, thus reducing compute time.
    \item Deployed a segmentation model (RetinaNet) for detecting car parts.
    \item Deployed a Mask-RCNN for marking the amount of visible damage on detected parts.
\end{itemize}
Technologies Used: \underline{Python, PyTorch, FastAI, Flask}

\divider

\cvevent{Google Summer of Code Fellow [2017 \& 2018] }{Advised by Dr. Jendrik Seipp, AI Group, University of Basel}{June -- Sep. 2017; May -- Aug. 2018}{Remote}
Worked on \href{https://github.com/jendrikseipp/vulture}{\underline{Vulture}}, a tool for detecting dead (i.e. unused) Python code.
\begin{itemize}
    \item Developed Vulture's Python API, allowing users to implement custom filters over results.
    \item Added support for Windows (Test on AppVeyor CI).
    \item Reduced false positives for projects using bindings from C/C++ (eg. PyQt -- count reduced from the prior 10,000 to zero).
    \item Wrote extensive tests, increasing Vulture's test coverage from an already excellent 95\% to 100\%.
    \item Switched to \textit{argparse} from \textit{optparse}.
    \item Performed an experimental integration with \textit{coverage} to automatically remove the false positives. (However, after careful consideration with the Vulture team, decided not to move on with it)
\end{itemize}

\faChain \, Work Report - \href{https://tinyurl.com/rahul-gsoc-2017}{\underline{2017}} \& \href{https://summerofcode.withgoogle.com/archive/2018/projects/6524402275450880}{\underline{2018}}

\cvsection{Competitions}

\cvevent{\textbf{S}tudents \textbf{A}utonomous Underwater \textbf{Ve}hicle Challenge (\textbf{SAVe})}{
\faTrophy \, Ranked 4th / 128 teams}{Jan. 25, 2019}{IIT Madras, India}

\begin{itemize}
    \item Lead Computer Team of 5 individuals to develop an Autonomous Underwater Vehicle (AUV).
    \item Developed multiple ROS packages for facilitation of 3D underwater movement, Perception, Sensor Fusion, Deep Learning based obstacle and object detection, etc.
\end{itemize}
Technologies Used: \underline{Python, C++, ROS, OpenCV, PyTorch}

\divider

\cvevent{ABU \textbf{Robocon} 2017}{\faTrophy \, Received "Jury's Choice Award"}{March 2017}{MIT Pune, India}

\begin{itemize}
    \item \underline{Ranked 21'st} among 120 participating teams.
    \item The task was to land the maximum number of \textit{Frisbees} on pads of varying heights.
    \item Developed control module for the robot.
\end{itemize}


% \medskip

% \cvsection{A Day of My Life}

% % Adapted from @Jake's answer from http://tex.stackexchange.com/a/82729/226
% % \wheelchart{outer radius}{inner radius}{
% % comma-separated list of value/text width/color/detail}
% \wheelchart{1.5cm}{0.5cm}{%
%   6/8em/accent!30/{Sleep,\\beautiful sleep},
%   3/8em/accent!40/Hopeful novelist by night,
%   8/8em/accent!60/Daytime job,
%       2/10em/accent/Sports and relaxation,
%   5/6em/accent!20/Spending time with family
% }

% \clearpage
% \cvsection[page2sidebar]{Publications}

% \nocite{*}

% \printbibliography[heading=pubtype,title={\printinfo{\faBook}{Books}},type=book]

% \divider

% \printbibliography[heading=pubtype,title={\printinfo{\faFileTextO}{Journal Articles}},type=article]

% \divider

% \printbibliography[heading=pubtype,title={\printinfo{\faGroup}{Conference Proceedings}},type=inproceedings]

% %% If the NEXT page doesn't start with a \cvsection but you'd
% %% still like to add a sidebar, then use this command on THIS
% %% page to add it. The optional argument lets you pull up the
% %% sidebar a bit so that it looks aligned with the top of the
% %% main column.
% % \addnextpagesidebar[-1ex]{page3sidebar}


\end{document}
